\documentclass[12pt]{article}
\begin{document}
\title{A Latex report on Future of Healthcare}
\maketitle{}

\section{Introduction}
Over the past 10 years, doom has been predicted almost continuously for our nations Academic Health Centers (AHCs). Perhaps the most important threat to AHCs has been the decrease in the perceived value of the patient care delivered by their doctors and hospitals: the payment differential to AHCs in comparison to community physicians and hospitals has virtually disappeared. The most immediate impact over the past 5 years has been a 30 percent reduction in direct payments for physicians in many AHCs.

The AHC has the tools and the personnel not only to improve patient care processes but also to understand how to decrease costs while maintaining quality. AHCs also have the size and expertise to establish control over geographic market share with services not available elsewhere. Such programs must be able to evolve and respond to market pressures, and the AHC must be an engine of innovation, continuously regenerating new knowledge and programs with Centers of Excellence and appropriate industry partnerships. Such progress is driven by better communication and greater sharing of information and collaboration at all levels, including building better physician referral networks. 
\section{Trends for Future of Healthcare}
1. More Patients\\
As we Baby Boomers age, the number of individuals arriving at age 65 will increase dramatically. Ten years from now, more patients will be living longer. The ability to treat patients with chronic disease such as heart disease is clearly lengthening their lives.

2. More Technology\\
As genetic diagnosis and treatment translate from cell to bedside, the information and armamentarium available to the clinician will increase perhaps inconceivably over the next 10 years. Markedly improved less invasive imaging along with less invasive treatment using catheter techniques will provide better functional outcomes with earlier resumption of activity. DNA chip technology or genetic fingerprinting will vastly improve risk assessment. Knowledge of the risks will increase the ability of other technology to extend life. Yet techniques such as these will require that we face and attempt to resolve a series of new ethical questions.

3. More Information\\
As the technology improves, the information deriving from patient care will also improve. With the Internet and its successors (which among other features will provide the important safeguards for confidentiality), the electronic medical record will not only be able to store patient information but also to provide information on best practice instantaneously, whether it is derived statistically from the practice of the physicians in that AHC, or based on health plan data or nationally generated practice guidelines. The opportunities for online clinical research are clear. The ability to question large numbers of patients and large segments of the general population may provide overall improved definitions of quality from the patient perspective.

4. Opportunity for Innovation\\
As care for many patients becomes more regularized and process and outcomes data become more similar, competition among practitioners will be based less on who has the best outcomes for common diseases and more on ability to innovate: developing the best care delivery models for patients with common diseases or developing new strategies for patients with uncommon diseases or presentations. Again, this will favor AHCs.

5. The Patient Will Be the Ultimate Consumer\\
As patients surf the web and as employers perhaps no longer choose the health plan for their employees (rather giving them a defined contribution to buy their own healthcare), patients will become the ultimate consumers. Measures of patient satisfaction and other patient-oriented report cards will assume increasing importance. An increasing consumer focus could reduce the need for wide geographic coverage of health plans that sell to employers: with the individual choosing the insurance product, patients will choose their own physicians and hospitals close to their own homes.

\section{Conclusion}
Over the next several years, strategies must be developed to ride out the decreased patient care revenue, increased uninsured, and increased competition on the basis of price and increased expense on technology.The benefits to riding out the storm will be an increased ability to demonstrate quality at a time when quality will be better understood, improved patient care and service at a time when patients will potentially be the direct consumers of healthcare, and clearly improved administrative systems that will be capable of handling larger numbers of patients with electronic medical records and billing systems. Such administrative capability, provided by investment in information systems, will form a major part of the strategy for AHCs over the next 10 years.

With decreasing margins in patient care and increasing numbers of uninsured, the physicians and administrators of the AHCs can become more effective and efficient in their practices – but this may not be enough: they can become important advocates on behalf of their patients for an improved healthcare system. Coverage for all is clearly beneficial to patients and physicians, allowing access and administrative simplification.














\end{document}